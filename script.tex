% Options for packages loaded elsewhere
\PassOptionsToPackage{unicode}{hyperref}
\PassOptionsToPackage{hyphens}{url}
%
\documentclass[
]{article}
\usepackage{amsmath,amssymb}
\usepackage{lmodern}
\usepackage{iftex}
\ifPDFTeX
  \usepackage[T1]{fontenc}
  \usepackage[utf8]{inputenc}
  \usepackage{textcomp} % provide euro and other symbols
\else % if luatex or xetex
  \usepackage{unicode-math}
  \defaultfontfeatures{Scale=MatchLowercase}
  \defaultfontfeatures[\rmfamily]{Ligatures=TeX,Scale=1}
\fi
% Use upquote if available, for straight quotes in verbatim environments
\IfFileExists{upquote.sty}{\usepackage{upquote}}{}
\IfFileExists{microtype.sty}{% use microtype if available
  \usepackage[]{microtype}
  \UseMicrotypeSet[protrusion]{basicmath} % disable protrusion for tt fonts
}{}
\makeatletter
\@ifundefined{KOMAClassName}{% if non-KOMA class
  \IfFileExists{parskip.sty}{%
    \usepackage{parskip}
  }{% else
    \setlength{\parindent}{0pt}
    \setlength{\parskip}{6pt plus 2pt minus 1pt}}
}{% if KOMA class
  \KOMAoptions{parskip=half}}
\makeatother
\usepackage{xcolor}
\usepackage[margin=1in]{geometry}
\usepackage{color}
\usepackage{fancyvrb}
\newcommand{\VerbBar}{|}
\newcommand{\VERB}{\Verb[commandchars=\\\{\}]}
\DefineVerbatimEnvironment{Highlighting}{Verbatim}{commandchars=\\\{\}}
% Add ',fontsize=\small' for more characters per line
\usepackage{framed}
\definecolor{shadecolor}{RGB}{248,248,248}
\newenvironment{Shaded}{\begin{snugshade}}{\end{snugshade}}
\newcommand{\AlertTok}[1]{\textcolor[rgb]{0.94,0.16,0.16}{#1}}
\newcommand{\AnnotationTok}[1]{\textcolor[rgb]{0.56,0.35,0.01}{\textbf{\textit{#1}}}}
\newcommand{\AttributeTok}[1]{\textcolor[rgb]{0.77,0.63,0.00}{#1}}
\newcommand{\BaseNTok}[1]{\textcolor[rgb]{0.00,0.00,0.81}{#1}}
\newcommand{\BuiltInTok}[1]{#1}
\newcommand{\CharTok}[1]{\textcolor[rgb]{0.31,0.60,0.02}{#1}}
\newcommand{\CommentTok}[1]{\textcolor[rgb]{0.56,0.35,0.01}{\textit{#1}}}
\newcommand{\CommentVarTok}[1]{\textcolor[rgb]{0.56,0.35,0.01}{\textbf{\textit{#1}}}}
\newcommand{\ConstantTok}[1]{\textcolor[rgb]{0.00,0.00,0.00}{#1}}
\newcommand{\ControlFlowTok}[1]{\textcolor[rgb]{0.13,0.29,0.53}{\textbf{#1}}}
\newcommand{\DataTypeTok}[1]{\textcolor[rgb]{0.13,0.29,0.53}{#1}}
\newcommand{\DecValTok}[1]{\textcolor[rgb]{0.00,0.00,0.81}{#1}}
\newcommand{\DocumentationTok}[1]{\textcolor[rgb]{0.56,0.35,0.01}{\textbf{\textit{#1}}}}
\newcommand{\ErrorTok}[1]{\textcolor[rgb]{0.64,0.00,0.00}{\textbf{#1}}}
\newcommand{\ExtensionTok}[1]{#1}
\newcommand{\FloatTok}[1]{\textcolor[rgb]{0.00,0.00,0.81}{#1}}
\newcommand{\FunctionTok}[1]{\textcolor[rgb]{0.00,0.00,0.00}{#1}}
\newcommand{\ImportTok}[1]{#1}
\newcommand{\InformationTok}[1]{\textcolor[rgb]{0.56,0.35,0.01}{\textbf{\textit{#1}}}}
\newcommand{\KeywordTok}[1]{\textcolor[rgb]{0.13,0.29,0.53}{\textbf{#1}}}
\newcommand{\NormalTok}[1]{#1}
\newcommand{\OperatorTok}[1]{\textcolor[rgb]{0.81,0.36,0.00}{\textbf{#1}}}
\newcommand{\OtherTok}[1]{\textcolor[rgb]{0.56,0.35,0.01}{#1}}
\newcommand{\PreprocessorTok}[1]{\textcolor[rgb]{0.56,0.35,0.01}{\textit{#1}}}
\newcommand{\RegionMarkerTok}[1]{#1}
\newcommand{\SpecialCharTok}[1]{\textcolor[rgb]{0.00,0.00,0.00}{#1}}
\newcommand{\SpecialStringTok}[1]{\textcolor[rgb]{0.31,0.60,0.02}{#1}}
\newcommand{\StringTok}[1]{\textcolor[rgb]{0.31,0.60,0.02}{#1}}
\newcommand{\VariableTok}[1]{\textcolor[rgb]{0.00,0.00,0.00}{#1}}
\newcommand{\VerbatimStringTok}[1]{\textcolor[rgb]{0.31,0.60,0.02}{#1}}
\newcommand{\WarningTok}[1]{\textcolor[rgb]{0.56,0.35,0.01}{\textbf{\textit{#1}}}}
\usepackage{graphicx}
\makeatletter
\def\maxwidth{\ifdim\Gin@nat@width>\linewidth\linewidth\else\Gin@nat@width\fi}
\def\maxheight{\ifdim\Gin@nat@height>\textheight\textheight\else\Gin@nat@height\fi}
\makeatother
% Scale images if necessary, so that they will not overflow the page
% margins by default, and it is still possible to overwrite the defaults
% using explicit options in \includegraphics[width, height, ...]{}
\setkeys{Gin}{width=\maxwidth,height=\maxheight,keepaspectratio}
% Set default figure placement to htbp
\makeatletter
\def\fps@figure{htbp}
\makeatother
\setlength{\emergencystretch}{3em} % prevent overfull lines
\providecommand{\tightlist}{%
  \setlength{\itemsep}{0pt}\setlength{\parskip}{0pt}}
\setcounter{secnumdepth}{-\maxdimen} % remove section numbering
\ifLuaTeX
  \usepackage{selnolig}  % disable illegal ligatures
\fi
\IfFileExists{bookmark.sty}{\usepackage{bookmark}}{\usepackage{hyperref}}
\IfFileExists{xurl.sty}{\usepackage{xurl}}{} % add URL line breaks if available
\urlstyle{same} % disable monospaced font for URLs
\hypersetup{
  pdftitle={Índice de Credibilidade para Bancos Centrais},
  pdfauthor={Victor Alves},
  hidelinks,
  pdfcreator={LaTeX via pandoc}}

\title{Índice de Credibilidade para Bancos Centrais}
\author{Victor Alves}
\date{2023-01-02}

\begin{document}
\maketitle

{
\setcounter{tocdepth}{2}
\tableofcontents
}
\hypertarget{carregando-pacotes}{%
\section{Carregando pacotes}\label{carregando-pacotes}}

\begin{Shaded}
\begin{Highlighting}[]
\FunctionTok{library}\NormalTok{(rbcb) }\CommentTok{\# dados para o Banco Central do Brasil}
\FunctionTok{library}\NormalTok{(tidyverse) }\CommentTok{\# manipulação e visualização de dados}
\FunctionTok{library}\NormalTok{(lubridate) }\CommentTok{\# manipulação de datas}
\FunctionTok{library}\NormalTok{(zoo)}
\FunctionTok{library}\NormalTok{(rvest)}
\FunctionTok{library}\NormalTok{(xml2)}
\FunctionTok{library}\NormalTok{(paletteer)}
\end{Highlighting}
\end{Shaded}

\hypertarget{limpeza-de-dados}{%
\section{Limpeza de dados}\label{limpeza-de-dados}}

\begin{Shaded}
\begin{Highlighting}[]
\FunctionTok{setwd}\NormalTok{(}\FunctionTok{dirname}\NormalTok{(rstudioapi}\SpecialCharTok{::}\FunctionTok{getActiveDocumentContext}\NormalTok{()}\SpecialCharTok{$}\NormalTok{path)) }\CommentTok{\# defininindo diretório como pasta onde o arquivo está}
\StringTok{\textasciigrave{}}\AttributeTok{\%not\_in\%}\StringTok{\textasciigrave{}} \OtherTok{\textless{}{-}}\NormalTok{ purrr}\SpecialCharTok{::}\FunctionTok{negate}\NormalTok{(}\StringTok{\textasciigrave{}}\AttributeTok{\%in\%}\StringTok{\textasciigrave{}}\NormalTok{) }\CommentTok{\# crianção de operador para negação de pertencimento}
\end{Highlighting}
\end{Shaded}

\hypertarget{dados-para-o-brasil}{%
\subsection{Dados para o Brasil}\label{dados-para-o-brasil}}

Para obtenção dos dados de inflação meta para o Brasil, foram utilzadas
as informações presentes no site
\href{https://www.bcb.gov.br/controleinflacao/historicometas}{Histórico
das metas para a inflação}, do BCB. Para isso, foi utilizada uma função
do Excel para obtenção de tabelas via HTML. O próximo passo aqui é
definir uma maneira de raspar automaticamente esta tabela via R.

\begin{Shaded}
\begin{Highlighting}[]
\CommentTok{\# inflação meta}
\NormalTok{brazil\_target }\OtherTok{\textless{}{-}}\NormalTok{ readxl}\SpecialCharTok{::}\FunctionTok{read\_xlsx}\NormalTok{(}\StringTok{\textquotesingle{}Dados/brazil\_target.xlsx\textquotesingle{}}\NormalTok{) }\SpecialCharTok{\%\textgreater{}\%} 
  \FunctionTok{select}\NormalTok{(}\SpecialCharTok{{-}}\FunctionTok{c}\NormalTok{(Norma, Data, }
            \StringTok{\textasciigrave{}}\AttributeTok{Tamanho do intervalo +/{-} (p.p.)}\StringTok{\textasciigrave{}}\NormalTok{, }\StringTok{\textasciigrave{}}\AttributeTok{Inflação efetiva (Variação do IPCA, \%)}\StringTok{\textasciigrave{}}\NormalTok{)) }\SpecialCharTok{\%\textgreater{}\%} 
  \FunctionTok{mutate}\NormalTok{(}
    \AttributeTok{Ano =} \FunctionTok{case\_when}\NormalTok{(Ano }\SpecialCharTok{==} \StringTok{\textquotesingle{}2003*\textquotesingle{}} \SpecialCharTok{\textasciitilde{}} \StringTok{\textquotesingle{}2003\textquotesingle{}}\NormalTok{,}
\NormalTok{              Ano }\SpecialCharTok{==} \StringTok{\textquotesingle{}2004*\textquotesingle{}} \SpecialCharTok{\textasciitilde{}} \StringTok{\textquotesingle{}2004\textquotesingle{}}\NormalTok{,}
              \ConstantTok{TRUE} \SpecialCharTok{\textasciitilde{}}\NormalTok{ Ano)) }\SpecialCharTok{\%\textgreater{}\%} 
  \FunctionTok{separate\_rows}\NormalTok{(}\StringTok{\textasciigrave{}}\AttributeTok{Meta (\%)}\StringTok{\textasciigrave{}}\NormalTok{, }\StringTok{\textasciigrave{}}\AttributeTok{Intervalo de tolerância (\%)}\StringTok{\textasciigrave{}}\NormalTok{, }\AttributeTok{convert =} \ConstantTok{TRUE}\NormalTok{, }\AttributeTok{sep =} \StringTok{\textquotesingle{}}\SpecialCharTok{\textbackslash{}n}\StringTok{\textquotesingle{}}\NormalTok{) }\SpecialCharTok{\%\textgreater{}\%} 
  \FunctionTok{separate}\NormalTok{( }
    \AttributeTok{col =} \StringTok{\textasciigrave{}}\AttributeTok{Intervalo de tolerância (\%)}\StringTok{\textasciigrave{}}\NormalTok{, }
    \AttributeTok{into =} \FunctionTok{c}\NormalTok{(}\StringTok{\textquotesingle{}inferior\textquotesingle{}}\NormalTok{, }\StringTok{\textquotesingle{}superior\textquotesingle{}}\NormalTok{), }
    \AttributeTok{sep =} \StringTok{\textquotesingle{}{-}\textquotesingle{}} 
\NormalTok{  ) }\SpecialCharTok{\%\textgreater{}\%} 
  \FunctionTok{rename}\NormalTok{(}\AttributeTok{meta =} \StringTok{\textasciigrave{}}\AttributeTok{Meta (\%)}\StringTok{\textasciigrave{}}\NormalTok{,}
         \AttributeTok{year =}\NormalTok{ Ano) }\SpecialCharTok{\%\textgreater{}\%} 
  \FunctionTok{mutate}\NormalTok{(}\AttributeTok{meta =} \FunctionTok{str\_replace}\NormalTok{(meta, }\StringTok{","}\NormalTok{, }\StringTok{"."}\NormalTok{),}
         \AttributeTok{inferior =} \FunctionTok{str\_replace}\NormalTok{(inferior, }\StringTok{","}\NormalTok{, }\StringTok{"."}\NormalTok{),}
         \AttributeTok{superior =} \FunctionTok{str\_replace}\NormalTok{(superior, }\StringTok{","}\NormalTok{, }\StringTok{"."}\NormalTok{)) }\SpecialCharTok{\%\textgreater{}\%} 
  \FunctionTok{mutate}\NormalTok{(}\AttributeTok{meta =} \FunctionTok{iconv}\NormalTok{(meta, }\StringTok{\textquotesingle{}utf{-}8\textquotesingle{}}\NormalTok{, }\StringTok{\textquotesingle{}ascii\textquotesingle{}}\NormalTok{, }\AttributeTok{sub=}\StringTok{\textquotesingle{}\textquotesingle{}}\NormalTok{),}
         \AttributeTok{inferior =} \FunctionTok{as.numeric}\NormalTok{(inferior),}
         \AttributeTok{superior =} \FunctionTok{as.numeric}\NormalTok{(superior),}
         \AttributeTok{year =} \FunctionTok{as.numeric}\NormalTok{(year))}
\end{Highlighting}
\end{Shaded}

Para obtenção das expectativas do mercado, foi utilizada uma função do
pacote `rbcb' para obter um dataframe para o relatório
FOCUS.\textbackslash{} Em primeiro momento se tomará apenas a inflação
corrente para o ano seguinte. Tendo também como objetivo para expansão o
cálculo com base na inflação esperada para os anos seguintes, se buscará
outras funções para obtenção destes dados.\textbackslash{} O dataframe
ainda possui um erro importante. Para os anos em que o FOCUS saiu muito
próximo do dia 31/12, a função de conversão de semanas para dia/mês/ano
está inserindo NA. Como forma de contornar este problema, será assumido
que a data não existente é o último dia do ano.

\begin{Shaded}
\begin{Highlighting}[]
\CommentTok{\# inflação esperada do Brasil }
\NormalTok{brazil\_forecast }\OtherTok{\textless{}{-}} \FunctionTok{get\_annual\_market\_expectations}\NormalTok{(}\StringTok{\textquotesingle{}IPCA\textquotesingle{}}\NormalTok{) }\SpecialCharTok{\%\textgreater{}\%} 
  \FunctionTok{filter}\NormalTok{(base }\SpecialCharTok{==} \DecValTok{0}\NormalTok{) }\SpecialCharTok{\%\textgreater{}\%} 
  \FunctionTok{mutate}\NormalTok{(}\AttributeTok{week =} \FunctionTok{week}\NormalTok{(}\FunctionTok{ymd}\NormalTok{(date)), }
         \AttributeTok{year =} \FunctionTok{year}\NormalTok{(}\FunctionTok{ymd}\NormalTok{(date)),}
         \AttributeTok{current =} \FunctionTok{case\_when}\NormalTok{(year }\SpecialCharTok{==}\NormalTok{ reference\_date }\SpecialCharTok{\textasciitilde{}} \DecValTok{1}\NormalTok{, }
                             \ConstantTok{TRUE} \SpecialCharTok{\textasciitilde{}} \DecValTok{0}\NormalTok{)) }\SpecialCharTok{\%\textgreater{}\%}
  \FunctionTok{filter}\NormalTok{(current }\SpecialCharTok{==} \DecValTok{1}\NormalTok{) }\SpecialCharTok{\%\textgreater{}\%} 
  \FunctionTok{group\_by}\NormalTok{(week, year, reference\_date) }\SpecialCharTok{\%\textgreater{}\%} 
  \FunctionTok{summarise}\NormalTok{(}\AttributeTok{median =} \FunctionTok{median}\NormalTok{(median)) }\SpecialCharTok{\%\textgreater{}\%} 
  \FunctionTok{mutate}\NormalTok{(}\AttributeTok{date =} \FunctionTok{as.Date}\NormalTok{(}\FunctionTok{paste}\NormalTok{(year, week, }\DecValTok{1}\NormalTok{, }\AttributeTok{sep=}\StringTok{\textquotesingle{}{-}\textquotesingle{}}\NormalTok{), }\StringTok{\textquotesingle{}\%Y{-}\%U{-}\%u\textquotesingle{}}\NormalTok{),}
         \AttributeTok{date =} \FunctionTok{replace\_na}\NormalTok{(date, }\FunctionTok{as.Date}\NormalTok{(}\FunctionTok{paste}\NormalTok{(year, }\DecValTok{12}\NormalTok{, }\DecValTok{31}\NormalTok{, }\AttributeTok{sep =} \StringTok{\textquotesingle{}{-}\textquotesingle{}}\NormalTok{)))) }
\end{Highlighting}
\end{Shaded}

\begin{verbatim}
## `summarise()` has grouped output by 'week', 'year'. You can override using the
## `.groups` argument.
\end{verbatim}

\begin{verbatim}
## Warning in strptime(x, format, tz = "GMT"): (0-based) yday 371 in year 2001 is
## invalid
\end{verbatim}

\begin{verbatim}
## Warning in strptime(x, format, tz = "GMT"): (0-based) yday 370 in year 2002 is
## invalid
\end{verbatim}

\begin{verbatim}
## Warning in strptime(x, format, tz = "GMT"): (0-based) yday 368 in year 2004 is
## invalid
\end{verbatim}

\begin{verbatim}
## Warning in strptime(x, format, tz = "GMT"): (0-based) yday 371 in year 2007 is
## invalid
\end{verbatim}

\begin{verbatim}
## Warning in strptime(x, format, tz = "GMT"): (0-based) yday 370 in year 2008 is
## invalid
\end{verbatim}

\begin{verbatim}
## Warning in strptime(x, format, tz = "GMT"): (0-based) yday 368 in year 2009 is
## invalid
\end{verbatim}

\begin{verbatim}
## Warning in strptime(x, format, tz = "GMT"): (0-based) yday 367 in year 2010 is
## invalid
\end{verbatim}

\begin{verbatim}
## Warning in strptime(x, format, tz = "GMT"): (0-based) yday 370 in year 2013 is
## invalid
\end{verbatim}

\begin{verbatim}
## Warning in strptime(x, format, tz = "GMT"): (0-based) yday 369 in year 2014 is
## invalid
\end{verbatim}

\begin{verbatim}
## Warning in strptime(x, format, tz = "GMT"): (0-based) yday 368 in year 2015 is
## invalid
\end{verbatim}

\begin{verbatim}
## Warning in strptime(x, format, tz = "GMT"): (0-based) yday 367 in year 2016 is
## invalid
\end{verbatim}

\begin{verbatim}
## Warning in strptime(x, format, tz = "GMT"): (0-based) yday 371 in year 2018 is
## invalid
\end{verbatim}

\begin{verbatim}
## Warning in strptime(x, format, tz = "GMT"): (0-based) yday 370 in year 2019 is
## invalid
\end{verbatim}

\begin{verbatim}
## Warning in strptime(x, format, tz = "GMT"): (0-based) yday 369 in year 2020 is
## invalid
\end{verbatim}

\begin{verbatim}
## Warning in strptime(x, format, tz = "GMT"): (0-based) yday 367 in year 2021 is
## invalid
\end{verbatim}

\begin{Shaded}
\begin{Highlighting}[]
\NormalTok{data\_brazil }\OtherTok{\textless{}{-}} \FunctionTok{right\_join}\NormalTok{(brazil\_target, brazil\_forecast, }\AttributeTok{by =} \StringTok{\textquotesingle{}year\textquotesingle{}}\NormalTok{) }\SpecialCharTok{\%\textgreater{}\%} 
  \FunctionTok{select}\NormalTok{(}\SpecialCharTok{{-}}\FunctionTok{c}\NormalTok{(year, week, reference\_date)) }\SpecialCharTok{\%\textgreater{}\%} 
  \FunctionTok{rename}\NormalTok{(}\AttributeTok{date =}\NormalTok{ date,}
         \AttributeTok{expectative =}\NormalTok{ median,}
         \AttributeTok{current =}\NormalTok{ meta) }\SpecialCharTok{\%\textgreater{}\%} 
  \FunctionTok{mutate}\NormalTok{(}\AttributeTok{country =} \StringTok{\textquotesingle{}Brasil\textquotesingle{}}\NormalTok{, }
         \AttributeTok{regime =} \StringTok{\textquotesingle{}interval\textquotesingle{}}\NormalTok{,}
         \AttributeTok{date =} \FunctionTok{as\_date}\NormalTok{(date, }\AttributeTok{format =} \StringTok{\textquotesingle{}\%Y{-}\%m{-}\%d\textquotesingle{}}\NormalTok{))}
\end{Highlighting}
\end{Shaded}

\hypertarget{dados-para-o-canaduxe1}{%
\subsection{Dados para o Canadá}\label{dados-para-o-canaduxe1}}

Os dados para a inflação meta do Canadá foram obtidos com base no site
\href{https://www.bankofcanada.ca/rates/indicators/key-variables/inflation-control-target/}{Inflation-Control
Target}.

\begin{Shaded}
\begin{Highlighting}[]
\CommentTok{\# inflação meta}
\NormalTok{canada\_target }\OtherTok{\textless{}{-}} \FunctionTok{read.csv}\NormalTok{(}\StringTok{\textquotesingle{}Dados/canada\_target.csv\textquotesingle{}}\NormalTok{, }
                          \AttributeTok{skip =} \DecValTok{19}\NormalTok{) }\SpecialCharTok{\%\textgreater{}\%} 
  \FunctionTok{rename}\NormalTok{(}\AttributeTok{inferior =}\NormalTok{ STATIC\_ATABLE\_CPILL,}
         \AttributeTok{superior =}\NormalTok{ STATIC\_ATABLE\_CPIHL,}
         \AttributeTok{current =}\NormalTok{ STATIC\_ATABLE\_V41690973) }\SpecialCharTok{\%\textgreater{}\%} 
  \FunctionTok{mutate}\NormalTok{(}\AttributeTok{date =} \FunctionTok{as.yearqtr}\NormalTok{(date,}
                           \AttributeTok{format =} \StringTok{"\%Y{-}\%m{-}\%d"}\NormalTok{))}
\end{Highlighting}
\end{Shaded}

Os dados para a inflação esperada do Canadá foram obtidos com base no
site
\href{https://www.bankofcanada.ca/rates/indicators/capacity-and-inflation-pressures/expectations/}{Expectations:
Definitions, Graphs and Data}. Esses dados precisarão ser revistos dado
que há alguns sinais de erros para ele.

\begin{Shaded}
\begin{Highlighting}[]
\CommentTok{\# inflação esperada}
\NormalTok{canada\_forecast }\OtherTok{\textless{}{-}} \FunctionTok{read.csv}\NormalTok{(}\StringTok{\textquotesingle{}Dados/canada\_forecast.csv\textquotesingle{}}\NormalTok{, }
                            \AttributeTok{skip =} \DecValTok{27}\NormalTok{, }\AttributeTok{na.strings =} \StringTok{""}\NormalTok{) }\SpecialCharTok{\%\textgreater{}\%} 
  \FunctionTok{select}\NormalTok{(date, INDINF\_EXPECTTWOTHREE\_Q) }\SpecialCharTok{\%\textgreater{}\%} 
  \FunctionTok{mutate}\NormalTok{(}\AttributeTok{date =} \FunctionTok{str\_replace}\NormalTok{(date, }\StringTok{"Q"}\NormalTok{, }\StringTok{" Q"}\NormalTok{),}
         \AttributeTok{date =} \FunctionTok{as.yearqtr}\NormalTok{(date)) }\SpecialCharTok{\%\textgreater{}\%} 
  \FunctionTok{rename}\NormalTok{(}\AttributeTok{expectative =}\NormalTok{ INDINF\_EXPECTTWOTHREE\_Q)}
\end{Highlighting}
\end{Shaded}

\begin{Shaded}
\begin{Highlighting}[]
\NormalTok{data\_canada }\OtherTok{\textless{}{-}} \FunctionTok{right\_join}\NormalTok{(canada\_target, canada\_forecast, }\AttributeTok{by =} \StringTok{\textquotesingle{}date\textquotesingle{}}\NormalTok{) }\SpecialCharTok{\%\textgreater{}\%} 
  \FunctionTok{separate}\NormalTok{(date, }\AttributeTok{into =} \FunctionTok{c}\NormalTok{(}\StringTok{\textquotesingle{}year\textquotesingle{}}\NormalTok{, }\StringTok{\textquotesingle{}quarter\textquotesingle{}}\NormalTok{), }\AttributeTok{sep =} \StringTok{\textquotesingle{} \textquotesingle{}}\NormalTok{) }\SpecialCharTok{\%\textgreater{}\%} 
  \FunctionTok{mutate}\NormalTok{(}
    \AttributeTok{month =} \FunctionTok{case\_when}\NormalTok{(quarter }\SpecialCharTok{==} \StringTok{\textquotesingle{}Q1\textquotesingle{}} \SpecialCharTok{\textasciitilde{}} \StringTok{\textquotesingle{}01/01\textquotesingle{}}\NormalTok{,}
\NormalTok{                      quarter }\SpecialCharTok{==} \StringTok{\textquotesingle{}Q2\textquotesingle{}} \SpecialCharTok{\textasciitilde{}} \StringTok{\textquotesingle{}01/04\textquotesingle{}}\NormalTok{,}
\NormalTok{                      quarter }\SpecialCharTok{==} \StringTok{\textquotesingle{}Q3\textquotesingle{}} \SpecialCharTok{\textasciitilde{}} \StringTok{\textquotesingle{}01/07\textquotesingle{}}\NormalTok{,}
\NormalTok{                      quarter }\SpecialCharTok{==} \StringTok{\textquotesingle{}Q4\textquotesingle{}} \SpecialCharTok{\textasciitilde{}} \StringTok{\textquotesingle{}01/12\textquotesingle{}}\NormalTok{,}
                      \ConstantTok{TRUE} \SpecialCharTok{\textasciitilde{}}\NormalTok{ quarter),}
    \AttributeTok{country =} \StringTok{\textquotesingle{}Canada\textquotesingle{}}\NormalTok{,}
    \AttributeTok{regime =} \StringTok{\textquotesingle{}interval\textquotesingle{}}\NormalTok{) }\SpecialCharTok{\%\textgreater{}\%} 
  \FunctionTok{unite}\NormalTok{(date, month, year, }\AttributeTok{sep =} \StringTok{"/"}\NormalTok{) }\SpecialCharTok{\%\textgreater{}\%} 
  \FunctionTok{select}\NormalTok{(}\SpecialCharTok{{-}}\FunctionTok{c}\NormalTok{(quarter)) }\SpecialCharTok{\%\textgreater{}\%} 
  \FunctionTok{mutate}\NormalTok{(}\AttributeTok{date =} \FunctionTok{as\_date}\NormalTok{(date, }\AttributeTok{format =} \StringTok{\textquotesingle{}\%d/\%m/\%Y\textquotesingle{}}\NormalTok{)) }\SpecialCharTok{\%\textgreater{}\%} 
  \FunctionTok{drop\_na}\NormalTok{()}
\end{Highlighting}
\end{Shaded}

\hypertarget{dados-para-a-argentina}{%
\subsection{Dados para a Argentina}\label{dados-para-a-argentina}}

Os dados para as expectativas de mercado foram obtidos com base no no
\href{https://www.bcra.gob.ar/publicacionesestadisticas/relevamiento_expectativas_de_mercado.asp}{Relevamiento
de Expectativas de Mercado (REM)}.\textbackslash{} A base aqui criada,
contará somente com a inflação esperada para o ano corrente, sendo que
na base original pode-se obter os valores para anos seguintes. Sendo
assim, serão buscadas formas de também se considerar estes valores
futuramente para o cálculo de um índice com base neles.\textbackslash{}

\begin{Shaded}
\begin{Highlighting}[]
\NormalTok{argentina\_forecast }\OtherTok{\textless{}{-}}\NormalTok{ readxl}\SpecialCharTok{::}\FunctionTok{read\_xlsx}\NormalTok{(}\StringTok{\textquotesingle{}Dados/argentina\_forecast.xlsx\textquotesingle{}}\NormalTok{, }
                                        \AttributeTok{sheet =} \DecValTok{2}\NormalTok{,}
                                        \AttributeTok{skip =} \DecValTok{1}\NormalTok{) }\SpecialCharTok{\%\textgreater{}\%} 
  \FunctionTok{select}\NormalTok{(}\SpecialCharTok{{-}}\FunctionTok{c}\NormalTok{(Promedio}\SpecialCharTok{:}\StringTok{\textasciigrave{}}\AttributeTok{Cantidad de participantes}\StringTok{\textasciigrave{}}\NormalTok{)) }\SpecialCharTok{\%\textgreater{}\%} 
  \FunctionTok{filter}\NormalTok{(Variable }\SpecialCharTok{==} \StringTok{\textquotesingle{}Precios minoristas (IPC nivel general; INDEC)\textquotesingle{}}\NormalTok{,}
\NormalTok{         Referencia }\SpecialCharTok{\%in\%} \FunctionTok{c}\NormalTok{(}\StringTok{\textquotesingle{}var. \% i.a.; dic{-}16\textquotesingle{}}\NormalTok{, }\StringTok{\textquotesingle{}var. \% i.a.; dic{-}17\textquotesingle{}}\NormalTok{, }
                           \StringTok{\textquotesingle{}var. \% i.a.; dic{-}18\textquotesingle{}}\NormalTok{, }\StringTok{\textquotesingle{}var. \% i.a.; dic{-}19\textquotesingle{}}\NormalTok{,}
                           \StringTok{\textquotesingle{}var. \% i.a.; dic{-}20\textquotesingle{}}\NormalTok{, }\StringTok{\textquotesingle{}var. \% i.a.; dic{-}21\textquotesingle{}}\NormalTok{,}
                           \StringTok{\textquotesingle{}var. \% i.a.; dic{-}22\textquotesingle{}}\NormalTok{, }\StringTok{\textquotesingle{}var. \% i.a.; dic{-}23\textquotesingle{}}\NormalTok{,}
                           \StringTok{\textquotesingle{}var. \% i.a.; dic{-}24\textquotesingle{}}\NormalTok{),}
\NormalTok{         Período }\SpecialCharTok{\%in\%} \FunctionTok{c}\NormalTok{(}\StringTok{\textquotesingle{}2016\textquotesingle{}}\NormalTok{, }\StringTok{\textquotesingle{}2017\textquotesingle{}}\NormalTok{, }\StringTok{\textquotesingle{}2018\textquotesingle{}}\NormalTok{, }\StringTok{\textquotesingle{}2019\textquotesingle{}}\NormalTok{,}
                        \StringTok{\textquotesingle{}2020\textquotesingle{}}\NormalTok{, }\StringTok{\textquotesingle{}2021\textquotesingle{}}\NormalTok{, }\StringTok{\textquotesingle{}2022\textquotesingle{}}\NormalTok{, }\StringTok{\textquotesingle{}2023\textquotesingle{}}\NormalTok{, }\StringTok{\textquotesingle{}2024\textquotesingle{}}\NormalTok{)) }\SpecialCharTok{\%\textgreater{}\%} 
  \FunctionTok{select}\NormalTok{(}\SpecialCharTok{{-}}\FunctionTok{c}\NormalTok{(Variable, Referencia)) }\SpecialCharTok{\%\textgreater{}\%} 
  \FunctionTok{rename}\NormalTok{(}\AttributeTok{date =} \StringTok{\textasciigrave{}}\AttributeTok{Fecha de pronóstico}\StringTok{\textasciigrave{}}\NormalTok{,}
         \AttributeTok{reference\_date =}\NormalTok{ Período,}
         \AttributeTok{expectative =}\NormalTok{ Mediana) }\SpecialCharTok{\%\textgreater{}\%} 
  \FunctionTok{mutate}\NormalTok{(}\AttributeTok{year =} \FunctionTok{year}\NormalTok{(}\FunctionTok{ymd}\NormalTok{(date)),}
         \AttributeTok{current =} \FunctionTok{case\_when}\NormalTok{(year }\SpecialCharTok{==}\NormalTok{ reference\_date }\SpecialCharTok{\textasciitilde{}} \DecValTok{1}\NormalTok{, }
                             \ConstantTok{TRUE} \SpecialCharTok{\textasciitilde{}} \DecValTok{0}\NormalTok{)) }\SpecialCharTok{\%\textgreater{}\%}
  \FunctionTok{filter}\NormalTok{(current }\SpecialCharTok{==} \DecValTok{1}\NormalTok{) }\SpecialCharTok{\%\textgreater{}\%} 
  \FunctionTok{select}\NormalTok{(}\SpecialCharTok{{-}}\FunctionTok{c}\NormalTok{(current, reference\_date))}
\end{Highlighting}
\end{Shaded}

Infelizmente não há uma base unificada para o regime de metas de
inflação para a Argetina. Portanto, será criado manualmente um dataframe
com base nos dados presentes em
\href{https://www.bcra.gob.ar/PoliticaMonetaria/Politica_Monetaria_i.asp}{Monetary
Policy}. Futuramente se buscará uma forma automatizada de se coletar
estes dados.

\begin{Shaded}
\begin{Highlighting}[]
\NormalTok{argentina\_target }\OtherTok{\textless{}{-}} \FunctionTok{tibble}\NormalTok{(}\AttributeTok{year =} \FunctionTok{c}\NormalTok{(}\DecValTok{2019}\NormalTok{, }\DecValTok{2020}\NormalTok{, }\DecValTok{2021}\NormalTok{, }\DecValTok{2022}\NormalTok{, }\DecValTok{2023}\NormalTok{, }\DecValTok{2024}\NormalTok{), }
                           \AttributeTok{current =} \FunctionTok{c}\NormalTok{(}\DecValTok{17}\NormalTok{, }\DecValTok{13}\NormalTok{, }\DecValTok{9}\NormalTok{, }\DecValTok{5}\NormalTok{, }\DecValTok{5}\NormalTok{, }\DecValTok{5}\NormalTok{),}
                           \AttributeTok{superior =} \FunctionTok{c}\NormalTok{(}\ConstantTok{NA}\NormalTok{, }\ConstantTok{NA}\NormalTok{, }\ConstantTok{NA}\NormalTok{, }\ConstantTok{NA}\NormalTok{, }\ConstantTok{NA}\NormalTok{, }\ConstantTok{NA}\NormalTok{),}
                           \AttributeTok{inferior =} \FunctionTok{c}\NormalTok{(}\ConstantTok{NA}\NormalTok{, }\ConstantTok{NA}\NormalTok{, }\ConstantTok{NA}\NormalTok{, }\ConstantTok{NA}\NormalTok{, }\ConstantTok{NA}\NormalTok{, }\ConstantTok{NA}\NormalTok{))}
\end{Highlighting}
\end{Shaded}

\begin{Shaded}
\begin{Highlighting}[]
\NormalTok{data\_argentina }\OtherTok{\textless{}{-}} \FunctionTok{right\_join}\NormalTok{(argentina\_target, argentina\_forecast, }\AttributeTok{by =} \StringTok{\textquotesingle{}year\textquotesingle{}}\NormalTok{) }\SpecialCharTok{\%\textgreater{}\%} 
  \FunctionTok{select}\NormalTok{(}\SpecialCharTok{{-}}\NormalTok{year) }\SpecialCharTok{\%\textgreater{}\%} 
  \FunctionTok{mutate}\NormalTok{(}\AttributeTok{country =} \StringTok{\textquotesingle{}Argentina\textquotesingle{}}\NormalTok{, }
         \AttributeTok{regime =} \StringTok{\textquotesingle{}target\textquotesingle{}}\NormalTok{)}
\end{Highlighting}
\end{Shaded}

\hypertarget{dados-para-o-chile}{%
\subsection{Dados para o Chile}\label{dados-para-o-chile}}

\hypertarget{dados-para-o-peru}{%
\subsection{Dados para o Peru}\label{dados-para-o-peru}}

\hypertarget{base-de-dados-final}{%
\section{Base de dados final}\label{base-de-dados-final}}

Para o cálculo do índice de credibilidade foi utilizado o índice
sugerido por
\href{https://econpapers.repec.org/paper/nbpnbpmis/209.htm}{Levieuge,
Lucotte, Ringuedé (2015)}.\textbackslash{} Sendo assim, assumindo que
valores abaixo da meta (ou intervalo da meta) geram perda de
credibilidade. O índice é construído tendo como condição o fato de o
Banco Central ter um regime com meta ou com intervalo de meta. Deste
modo, sendo \(\pi^e\) a inflação esperada, \(\bar{\pi}\) a inflação meta
e \(\pi^{\min} \pi^{\max}\) os limites do intervalo da meta.

\begin{itemize}
\tightlist
\item
  Caso o Banco Central tenha um valor fixo de inflação meta:
\end{itemize}

\(\text{credibility} = \dfrac{1}{exp(\pi^e - \bar{\pi}) - (\pi^e - \bar{\pi})}\)
* Caso o Banco Central tenha um valor em intervalo de inflação meta:

\(\text{credibility} = \begin{cases} \dfrac{1}{exp(\pi^e - \bar{\pi^{\min}}) - (\pi^e - \bar{\pi}^{\min})}, & \text{se } \pi^e<\bar{\pi}^{\min}\\ 1, & \text{se } \pi \in \left[\bar{\pi}^\min, \bar{\pi}^\max \right]\\ \dfrac{1}{exp(\pi^e - \bar{\pi^{\max}}) - (\pi^e - \bar{\pi}^{\max})}, & \text{se } \pi^e>\bar{\pi}^{\max} \end{cases}\)

\begin{Shaded}
\begin{Highlighting}[]
\NormalTok{database }\OtherTok{\textless{}{-}} \FunctionTok{rbind}\NormalTok{(data\_brazil, data\_canada, data\_argentina) }\SpecialCharTok{\%\textgreater{}\%} 
    \FunctionTok{mutate}\NormalTok{(}
    \AttributeTok{current =} \FunctionTok{as.numeric}\NormalTok{(current),}
    \AttributeTok{credibility =} \FunctionTok{case\_when}\NormalTok{(}
\NormalTok{      regime }\SpecialCharTok{==} \StringTok{\textquotesingle{}interval\textquotesingle{}} \SpecialCharTok{\&}\NormalTok{ expectative }\SpecialCharTok{\textless{}}\NormalTok{ inferior }\SpecialCharTok{\textasciitilde{}} \DecValTok{1}\SpecialCharTok{/}\NormalTok{(}\FunctionTok{exp}\NormalTok{(expectative }\SpecialCharTok{{-}}\NormalTok{ inferior) }\SpecialCharTok{{-}}\NormalTok{ (expectative }\SpecialCharTok{{-}}\NormalTok{ inferior)),}
\NormalTok{      regime }\SpecialCharTok{==} \StringTok{\textquotesingle{}interval\textquotesingle{}} \SpecialCharTok{\&}\NormalTok{ expectative }\SpecialCharTok{\textgreater{}=}\NormalTok{ inferior }\SpecialCharTok{\&}\NormalTok{ expectative }\SpecialCharTok{\textless{}=}\NormalTok{ superior }\SpecialCharTok{\textasciitilde{}} \DecValTok{1}\NormalTok{,}
\NormalTok{      regime }\SpecialCharTok{==} \StringTok{\textquotesingle{}interval\textquotesingle{}} \SpecialCharTok{\&}\NormalTok{ expectative }\SpecialCharTok{\textgreater{}}\NormalTok{ superior }\SpecialCharTok{\textasciitilde{}} \DecValTok{1}\SpecialCharTok{/}\NormalTok{(}\FunctionTok{exp}\NormalTok{(expectative }\SpecialCharTok{{-}}\NormalTok{ superior) }\SpecialCharTok{{-}}\NormalTok{ (expectative }\SpecialCharTok{{-}}\NormalTok{ superior)),}
\NormalTok{      regime }\SpecialCharTok{==} \StringTok{\textquotesingle{}target\textquotesingle{}} \SpecialCharTok{\textasciitilde{}} \DecValTok{1}\SpecialCharTok{/}\NormalTok{(}\FunctionTok{exp}\NormalTok{(expectative }\SpecialCharTok{{-}}\NormalTok{ current) }\SpecialCharTok{{-}}\NormalTok{ (expectative }\SpecialCharTok{{-}}\NormalTok{ current)) }
\NormalTok{    )}
\NormalTok{  )}
\end{Highlighting}
\end{Shaded}

\begin{Shaded}
\begin{Highlighting}[]
\NormalTok{database }\SpecialCharTok{\%\textgreater{}\%} 
  \FunctionTok{ggplot}\NormalTok{() }\SpecialCharTok{+} 
  \FunctionTok{geom\_line}\NormalTok{(}\FunctionTok{aes}\NormalTok{(date, credibility)) }\SpecialCharTok{+}
  \FunctionTok{scale\_x\_date}\NormalTok{(}\AttributeTok{limits =} \FunctionTok{as.Date}\NormalTok{(}\FunctionTok{c}\NormalTok{(}\StringTok{"2001{-}12{-}12"}\NormalTok{, }\StringTok{"2022{-}12{-}30"}\NormalTok{))) }\SpecialCharTok{+}
  \FunctionTok{labs}\NormalTok{(}\AttributeTok{title =} \StringTok{\textquotesingle{}Índice de credibilidade do Banco Central\textquotesingle{}}\NormalTok{,}
       \AttributeTok{subtitle =} \StringTok{\textquotesingle{}Dados semanais para o Brasil com base no boletim FOCUS\textquotesingle{}}\NormalTok{,}
       \AttributeTok{x =} \StringTok{\textquotesingle{}Data\textquotesingle{}}\NormalTok{,}
       \AttributeTok{y =} \StringTok{\textquotesingle{}Índice de credibilidade\textquotesingle{}}\NormalTok{,}
       \AttributeTok{caption =} \StringTok{\textquotesingle{}Baseado em Levieuge, Lucotte, Ringuedé (2015)\textquotesingle{}}\NormalTok{,}
       \AttributeTok{color =} \StringTok{\textquotesingle{}País\textquotesingle{}}\NormalTok{) }\SpecialCharTok{+}
  \FunctionTok{theme\_bw}\NormalTok{() }\SpecialCharTok{+}
  \FunctionTok{theme}\NormalTok{(}
    \AttributeTok{plot.title=}\FunctionTok{element\_text}\NormalTok{(}\AttributeTok{size=}\DecValTok{16}\NormalTok{, }\AttributeTok{face=}\StringTok{"bold"}\NormalTok{, }\AttributeTok{hjust =} \FloatTok{0.5}\NormalTok{),}
    \AttributeTok{legend.text=}\FunctionTok{element\_text}\NormalTok{(}\AttributeTok{size=}\DecValTok{10}\NormalTok{),}
    \AttributeTok{axis.text=}\FunctionTok{element\_text}\NormalTok{(}\AttributeTok{size=}\DecValTok{10}\NormalTok{),}
    \AttributeTok{axis.title =} \FunctionTok{element\_text}\NormalTok{(}\AttributeTok{size =} \DecValTok{12}\NormalTok{),}
    \AttributeTok{legend.title =} \FunctionTok{element\_text}\NormalTok{(}\AttributeTok{size =} \DecValTok{10}\NormalTok{)) }\SpecialCharTok{+}
  \FunctionTok{scale\_color\_grey}\NormalTok{() }\SpecialCharTok{+}
  \FunctionTok{facet\_wrap}\NormalTok{(}\FunctionTok{vars}\NormalTok{(country), }\AttributeTok{ncol =} \DecValTok{2}\NormalTok{)}
\end{Highlighting}
\end{Shaded}

\begin{verbatim}
## Warning: Removed 100 row(s) containing missing values (geom_path).
\end{verbatim}

\includegraphics{script_files/figure-latex/unnamed-chunk-13-1.pdf}

\end{document}
